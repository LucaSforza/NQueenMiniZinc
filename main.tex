\documentclass{article}
\usepackage{amsmath}

\begin{document}

\section{NQueen MiniZinc}

\subsection{Modellazione}

Sia $n > 3$ la dimensione del problema.

\begin{equation}
    X = \{Q_1, \dots, Q_n\}
\end{equation}

Ogni variabile $Q_i$ rapprensenta la posizione della regina
sull'$i$-esima colonna.
Perciò viene naturale definire i domini.

\begin{equation}
    D_i = \{ x \mid 1 \leq x \leq n \}
\end{equation}

Questi sono i domini per l'$i$-esima variabile.

\begin{multline}
    \forall i,j \mid Q_i,Q_j \in X \land i \not = j \quad \\
    C_{(i,j)} = 
    (
        \{Q_i,Q_j\}, \{(x,y) \mid x \in D_i \land y \in D_j \land x \not = y \land |x - y| \not = |i - j|\}
    )
\end{multline}

I vincoli sono fatti in questo modo. Per ogni coppia di variabile (quindi per ogni colonna)
le regine non possono stare sulla stessa riga ($x \not = y$) e sulla stessa diagonale (il fatto che non sono sulla stessa colonna è implicito).

Quindi per verificare che due regine non siano sulla stessa diagonale abbiamo che il valore assoluto della differenza sulla posizione delle regine sulle righe deve essere diverso dal valore assoluto della
differenza sulla posizione delle regine sulle colonne.

Da notare però che ho dei vincoli duplicati, per esempio nel caso $n = 5$ avrò il vincolo $C_{(1,2)} = C_{(2,1)}$.

Avere vincoli in piu' non è necessariamente un problema.
Però quando verrà implementato con MiniZinc questi vincoli in piu' daranno problemi per l'efficienza.

Quindi ecco una formulazione equivalente dove elimino i vincoli duplicati:

\begin{multline}
    \forall i,j \mid Q_i,Q_j \in X \land j < i \quad \\
    C_{(i,j)} = 
    (
        \{Q_i,Q_j\}, \{(x,y) \mid x \in D_i \land y \in D_j \land x \not = y \land |x - y| \not = i - j\}
    )
\end{multline}

\subsection{Istanziazione}

Per il caso $n = 5$ abbiamo che ne variabili sono:

\begin{equation}
    X = \{Q_1, Q_2, Q_3, Q_4, Q_5\}
\end{equation}

I domini sono:

\begin{equation}
    \forall i \mid 1 \leq i \leq n = 5, D_i = \{1,2,3,4,5\}
\end{equation}

\newpage

Invece i vincoli sono:
\[
\begin{cases}
    C_{(2,1)} = (\{Q_2,Q_1\}, \{(2,1), (3,1), (4,1), (5,1), (1,3), (2,4), (3,5)\}) \\
    C_{(3,1)} = (\{Q_3,Q_1\}, \{(2,1), (3,1), (4,1), (5,1), (1,4), (2,5)\}) \\
    C_{(4,1)} = (\{Q_4,Q_1\}, \{(2,1), (3,1), (4,1), (5,1), (1,5)\}) \\
    C_{(5,1)} = (\{Q_5,Q_1\}, \{(2,1), (3,1), (4,1), (5,1)\}) \\
    C_{(3,2)} = (\{Q_3,Q_2\}, \{(1,2), (3,2), (4,2), (5,2), (2,4), (3,5)\}) \\
    C_{(4,2)} = (\{Q_4,Q_2\}, \{(1,2), (3,2), (4,2), (5,2), (2,5)\}) \\
    C_{(5,2)} = (\{Q_5,Q_2\}, \{(1,2), (3,2), (4,2), (5,2)\}) \\
    C_{(4,3)} = (\{Q_4,Q_3\}, \{(1,3), (2,3), (4,3), (5,3), (3,5)\}) \\
    C_{(5,3)} = (\{Q_5,Q_3\}, \{(1,3), (2,3), (4,3), (5,3)\}) \\
    C_{(5,4)} = (\{Q_5,Q_4\}, \{(1,4), (2,4), (3,4), (5,4)\}) 
\end{cases}
\]

\subsection{Implementazione MiniZinc}

Si può trovare l'implementazione della modellazione descritta in questo documento nella
stesso repository nel file model.mzn. Per le costanti guardare data.dzn.

\end{document}